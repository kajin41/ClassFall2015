\documentclass[10pt,a4paper]{article}
\usepackage[utf8]{inputenc}
\usepackage{amsmath}
\usepackage{amsfonts}
\usepackage{amssymb}
\begin{document}
\title{Differential Equations\\Part 2}
\date{}
\maketitle
\pagebreak
\date{Wednesday, October 21}
\section*{Laplace Transforms}
Chapter 7
\subsection*{Definition}
Let f(t) be an integrable function on\begin{equation} [0,\infty)\end{equation}. The Laplace transform of f(t) is defined by
\begin{equation}
\mathcal{L}(f)=F(s)=\int_{0}^{\infty}f(t)e^{-st}dt
\end{equation}
and
\begin{equation}
\mathcal{L}(f(t))=F(s)
\end{equation}
if and only if 
\begin{equation}
\mathcal{L}^{-1}(F(s))=f(t)
\end{equation}
\begin{align}
\mathcal{L}(1)&=\int _{0} ^{\infty} e^{-st}dt=\frac{-e^{-st}}{s}|_{0}^{\infty} =^{s>0} \frac{-1}{s}(0-1)=\frac{1}{s}\\
&=\frac{-1}{s}(0-0)-\frac{e^{-st}}{s^{2}}|_{0}^{\infty}=\frac{-1}{s^{2}}(0-1)=\frac{1}{s^{2}}
\end{align}
\subsection*{Inverse Laplace Transform}
\begin{align*}
Given\ F(s).\\
\mathcal{L}^{-1}(F(s))&=f(t)\ if\ and\ only\ if\\
\mathcal{L}(f(t0)&=F(s)\\\\
\textbf{EX.}\ &Find\ \mathcal{L}^{-1}(\frac{1}{s}), \mathcal{L}^{-1}(\frac{1}{s^{2}+a^{2}})\\
&and\ \mathcal{L}^{-1}(\frac{1}{s^{n}}), n\geq1\\
Height = (\frac{1}{c}xQ_{in}xtime
\end{align*}

\end{document}